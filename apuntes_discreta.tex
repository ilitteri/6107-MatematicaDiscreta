\documentclass[]{article}

%\usepackage[utf-8]{inputenc}
\usepackage{graphicx} % Required for including pictures
\usepackage[figurename=Figure]{caption}
\usepackage{float}    % For tables and other floats
\usepackage{verbatim} % For comments and other
\usepackage{amsmath}  % For math
\usepackage{amssymb}  % For more math
\usepackage{fullpage} % Set margins and place page numbers at bottom center
%\usepackage{paralist} % paragraph spacing
\usepackage{listings} % For source code
\usepackage{subfig}   % For subfigures
%\usepackage{physics}  % for simplified dv, and 
\usepackage{enumitem} % useful for itemization
\usepackage{siunitx}  % standardization of si units

\usepackage{tikz,bm} % Useful for drawing plots
%\usepackage{tikz-3dplot}
\usepackage{circuitikz}
\usepackage{dsfont}
\usepackage{hyperref}
\usepackage[T1]{fontenc}
\usepackage{multirow}


\usepackage{amsthm}


%opening
\title{Notas de Matematica Discreta}
\author{Ivan Litteri}
\date{}
\begin{document}

\maketitle

\section{Lógica}\label{sec:logica}

Las reglas de la logica le dan un significado preciso a los enunciados matematicos o sentensias matematicas. Estas reglas se usan para distinguir entre argumentos validos y no validos. La logica tiene ademas numerosas aplicaciones en ciencias de la computacion. Las reglas de la logiuca se usan en el diseño de circuitos de ordenador, la construccion de programas informaticos, la verificacion de que un programa esta bien construido y mas.

\subsection{Logica Proposicional}

\subsubsection*{Proposicion}
Una proposicion es una oracion declarativa que es correcta o falsa, pero no ambas cosas a la vez.

\subsubsection*{Ejemplos}

\begin{enumerate}
	\item Todas las siguientes oraciones declarativas son proposiciones:
		\begin{enumerate}
			\item Bruselas es la capital de la Union Europea.
			\item Toronto es la capital de Canada.
			\item 1 + 1 = 2.
			\item 2 + 2 = 3.
		\end{enumerate}
	Las proposiciones 1 y 3 son correctas, mientras que la 2 y 4 son falsas.
	
	\item Considera las siguientes oraciones:
		\begin{enumerate}
			\item ¿Que hora es?
			\item Lee esto con atencion.
			\item x + 1 = 2.
			\item x + y = z.
		\end{enumerate}
	Las frases 1 y 2 no son proposiciones porque no son declarativas. Las frases 3 y 4 no son proposiciones porque no son ni verdaderas ni falsas, ya que no se les han asignado valores a las variables. 
\end{enumerate}

El \textbf{valor de verdad} de una proposicion es $V$ si es verdadera y $F$ si es falsa.

Se llama \textbf{calculo proposicional} o \textbf{logica proposicional} al area de la logica que trata de proposiciones. Las nuevas proposiciones, llamadas \textbf{proposicion compuestas} o \textbf{proposiciones compuestas}, se forman a partir de las existentes usando operadores logicos.

\subsubsection*{Tabla de Verdad}

Una \textbf{tabla de verdad} muestra las relaciones entre los valores de verdad ed proposiciones. Son especialmente valiosas a la hora de determinar los valores de verdad de proposiciones construidas a partir de proposiciones simples.

\subsubsection*{Proposiciones Compuestas}

\textbf{Definicion.}  Sea $p$ una proposicion. El enunciado
		\begin{center}
			\textit{No se cumple $p$},
		\end{center}
	es otra proposicion, llamada la \textit{negacion} de $p$. La negacion de $p$ se denota mediante $p'$ o ($\not p$ o $\neg p$). La proposicion $p'$ se lee \textit{no p}.\\

	La negacion de una proposicion se puede considerar como el resultado de aplicar el \textbf{operador negacion} sobre una proposicion. El operador negacion construye una nueva proposicion a partir de la proposicion individual existente. 

	\begin{table}[H]
		\begin{center}
			\label{tab:and}
			\begin{tabular}{c|c}
				$p$ & $p'$ \\
				\hline
				0 & 1 \\
				1 & 0 
			\end{tabular}
		\end{center}
	\end{table}

\textbf{Definicion.} Sean $p$ y $q$ proposiciones. La proposicion \textit{p y q}, denotada por $p \wedge q$, es la proposicion que es verdadera cuando tanto $p$ como $q$ son verdaderas y falsa en cualquier otro caso. La proposicion $p \wedge q$ se llama \textit{conjuncion} de \textit{p y q}.\\

	\begin{table}[H]
		\begin{center}
			\label{tab:and}
			\begin{tabular}{cc|c}
				$p$ & $q$ & $p \wedge q$ \\
				\hline
				0 & 0 & 0 \\
				0 & 1 & 0 \\
				1 & 0 & 0 \\
				1 & 1 & 1 \\
			\end{tabular}
		\end{center}
	\end{table}

\textbf{Definicion.} Sean $p$ y $q$ proposiciones. La proposicion \textit{p o q}, denotada por $p \vee q$, es la proposicion que es falsa cuando tanto $p$ como $q$ son falsas y verdadera en cualquier otro caso. La proposicion $p \vee q$ se llama \textit{disyuncion} de \textit{p o q}.\\

	\begin{table}[H]
		\begin{center}
			\label{tab:or}
			\begin{tabular}{cc|c}
				$p$ & $q$ & $p \vee q$ \\
				\hline
				0 & 0 & 0 \\
				0 & 1 & 1 \\
				1 & 0 & 1 \\
				1 & 1 & 1 \\
			\end{tabular}
		\end{center}
	\end{table}

\textbf{Definicion.} Sean $p$ y $q$ proposiciones. El conectivo logico \textit{o exclusivo} de $p$ y $q$, denotada por $p \oplus q$, es la proposicion que es verdadera cuando exactamente una de las proposiciones $p$ y $q$ es verdadera y falsa en cualquier otro caso.\\

	\begin{table}[H]
		\begin{center}
			\label{tab:xor}
			\begin{tabular}{cc|c}
				$p$ & $q$ & $p \oplus q$ \\
				\hline
				0 & 0 & 0 \\
				0 & 1 & 1 \\
				1 & 0 & 1 \\
				1 & 1 & 0 \\
			\end{tabular}
		\end{center}
	\end{table}

\subsubsection*{Implicaciones}

El concepto matematico de implicacion es independiente de la relacion causa-efecto entre hipotesis y conclusion. Especifica valores de verdad, no se basa en el uso del lenguaje.\\

\textbf{Definicion.} Sean $p$ y $q$ proposiciones. La \textit{implicacion} $p \rightarrow q$ es la proposicion que es falsa cuando $p$ es verdadera y $q$ es falsa, y verdadera en cualquier otro caso. En esta implicacion $p$ se llama \textit{hipotesis} o \textit{antecedente} o \textit{premisa} y $q$ se llama \textit{tesis} o \textit{conclusion} o \textit{consecuencia}.\\

	\begin{table}[H]
		\begin{center}
			\label{tab:then}
			\begin{tabular}{cc|c}
				$p$ & $q$ & $p \rightarrow q$ \\
				\hline
				0 & 0 & 1 \\
				0 & 1 & 1 \\
				1 & 0 & 0 \\
				1 & 1 & 1 \\
			\end{tabular}
		\end{center}
	\end{table}

	Las declaraciones condicionales (como las implicaciones) tienen un rol esencial en el razonamiento matematico, existe una variedad de terminologias para expresar $p \rightarrow q$. Entre ellas:

	\begin{table}[H]
		\begin{center}
			\begin{tabular}{c c}
				"si $p$, entonces $q$" & "$p$ implica $q$"\\
				"si $p$, $q$" & "$p$ solo si $q$"\\
				"$p$ es suficiente para $q$" & "una condicion suficiente para $q$ es $p$"\\
				"$q$ si $p$" & "$q$ cuando sea $p$"\\
				"$q$ cuando $p$" & "$q$ es necesario para $p$"\\
				"una condicion necesaria para $p$ es $q$" & "$q$ sigue de $p$"\\
				"$q$ a menos que $p$" & "$q$ siempre que $p$"\\
			\end{tabular}
		\end{center}
	\end{table}

\subsubsection*{Recipropca, contrarreciproca e inversa}

Hay algunas implicaciones relacionadas con $p \rightarrow q$ que pueden formarse a partir de ella. La proposicion $q \rightarrow p$ se llama \textbf{reciproca} de $p \rightarrow q$. La \textbf{contrarreciproca} de $p \rightarrow q$ es $q' \rightarrow p'$. La proposicion $p' \rightarrow q'$ es la \textbf{inversa} de $p \rightarrow q$. 
Cuando dos proposiciones compuestas tienen siempre los mismos valores de verdad las llamamos \textbf{equivalentes}, de tal forma que una implicacion y su contrarreciproca son equivalentes. La reciproca y la inversa de una implicacion tambien son equivalentes.\\

\textbf{Definicion.} Sean $p$ y $q$ proposiciones. La \textit{bicondicional}, o \textit{doble impliacion}, $p \leftrightarrow q$ es la proposicion que es verdadera cuando $p$ y $q$ tienen los mismos valores de veradd y falsa en los otros casos.\\

	\begin{table}[H]
		\begin{center}
			\label{tab:then}
			\begin{tabular}{cc|c}
				$p$ & $q$ & $p \leftrightarrow q$ \\
				\hline
				0 & 0 & 1 \\
				0 & 1 & 0 \\
				1 & 0 & 0 \\
				1 & 1 & 1 \\
			\end{tabular}
		\end{center}
	\end{table}

Notese que $p \leftrightarrow q$ es verdadera cuando $p \rightarrow q$ y $q \rightarrow p$ son verdaderas y falso de otra manera. Las formas mas comunes de expresar esto es:

	\begin{table}[H]
		\begin{center}
			\begin{tabular}{c}
				"$p$ es necesario y suficiente para $q$"\\
				"si $p$ entonces $q$, y biceversa"\\
				"$p$ sii $q$". "$p$ exactamente cuando $q$"
			\end{tabular}
		\end{center}
	\end{table}

\subsubsection*{Precedencia de operadores logicos}\label{sec:precedenda}

\begin{table}[H]
	\begin{center}
		\label{tab:precedencia}
		\begin{tabular}{c|c}
		Operador & Precedencia \\
		\hline
		' (not) & 1 \\
		$\wedge$ & 2 \\
		$\vee$ & 4 \\
		$\rightarrow$ & 4 \\
		$\leftrightarrow$ & 5 \\
		\end{tabular}
	\end{center}
\end{table}

\subsubsection*{Logica y Operaciones con Bits}\label{sec:operaciones-bits}

Un \textbf{bit} tiene dos valores posibles: 0 y 1. La palabra bit viene de la expresion inglesa \textit{binary digit}. Un bit se puede utilizar para represenar un valor de verdad. Se usa 1 para representar V de verdadero y 0 para representar F de falso. Una variable se llama \textbf{variable booleana} si su valor es verdadero o falso. Se puede representar una variable booleana con bits.\\

\textbf{Definicion}. Una \textit{cadena de bits} es una sucesion de cero o mas bits. La longitud de esta cadena es el numero de bits de la cadena.

\subsection{Aplicaciones de la Logica Proposicional}

\subsubsection*{Traducir Oraciones}

Traducir oraciones del lenguaje natural a expresiones legicas es una parte esencial de especificar sistemas tantohardware como sofware.

Las especificaciones de sistema deben ser \textbf{consistentes}, esto es, no deben contener recursos conflictivos requisitos que podrían utilizarse para derivar una contradicción. Cuando las especificaciones no son consistentes, no habría forma de desarrollar un sistema que satisfaga todas las especificaciones.

\subsubsection*{Busquedas Booleanas}

En las busquedas booleanas se usa la conexion $AND$ para emparejar datos almacenados que contengan los dos terminos de la busqueda, la conexion $OR$ se usa para emparejar uno o ambos terminos de la busqueda y la conexion $NOT$ (a veces escrita $AND NOT$) se usa para excluir un termino particular de busqueda.

\subsubsection*{Circuitos Logicos}

Un \textbf{circuito logico} (o \textbf{circuito digital}) recibe señales de entrada $p_{1}, ..., p_{n}$, cada bit [0 (off) o 1 (on)], y produce una señal de salida $s_{1}, ..., s_{n}$ para cada bit.

Circuitos digitales complejos pueden contruirse a partir de tres circuitos basicos llamados \textbf{compuertas}. El \textbf{inversor}, o \textbf{compuerta NOT}, toma el bit de entrada $p$, y produce una salida $p'$. La \textbf{compuerta OR} toma dos entradas $p$ y $q$, cada una un bit, y produce una señal de salida $p \vee q$. Finalmente, la \textbf{compuerta AND} toma dos entradas $p$ y $q$, cada una un bit, y produce una señal de salida $p \wedge q$.

\subsection{Equivalencias proposicionales}

\textbf{Definicion.} Una proposicion compuesta que es siempre verdadera, no importa los valores de verdad de las proposiciones que la componen, se denomina \textit{tautologia}. Una proposicion compuesta que es siepre falsa se denomina \textit{contradiccion}. Finalmente, una proposicion que no es ni una tautologia ni una contradiccion se denomina \textit{contingencia}.

\subsubsection*{Equivalencias Logicas}

Las proposicion compuestas que tienen los mismos valores de verdad en todos los casos posibles se llaman \textbf{logicamente equivalentes}.\\

\textbf{Definicion.} Se dice que las proposiciones $p$ y $q$ son \textit{logicamente equivalentes} si $p \leftrightarrow q$ que es una tautologia. La notacion $p \equiv q$ denota que $p$ y $q$ son logicamente equivalentes.

\begin{table}[H]
	\caption*{\textbf{Equivalencias Logicas}}
	\parbox{.5\linewidth}{
	\begin{center}
		\label{tab:equivalencias-logicas}
		\begin{tabular}{|c|c|}
			\hline
			\textit{Equivalencia} & \textit{Nombre} \\
			\hline
			$p\wedge V \equiv p$ & \multirow{2}{*}{Leyes de identidad} \\
			$p\vee F \equiv p$ & \\
			\hline
			$p\vee V \equiv V$ & \multirow{2}{*}{Leyes de dominacion} \\
			$p\wedge F \equiv F$ & \\
			\hline
			$p\vee p \equiv p$ & \multirow{2}{*}{Leyes de idempotentes} \\
			$p\wedge p \equiv p$ & \\
			\hline
			$(p')' \equiv p$ & Ley de la doble negacion\\
			\hline
			$p\vee q \equiv q\vee p$ & \multirow{2}{*}{Leyes de conmutativas} \\
			$p\wedge q \equiv q\wedge p$ & \\
			\hline
			$(p\vee q)\vee r \equiv p \vee(q\vee p)$ & \multirow{2}{*}{Leyes de asociativas} \\
			$(p\wedge q)\wedge r \equiv p \wedge(q\wedge p)$ & \\
			\hline
			$p \vee (q \wedge r) \equiv (p\vee q) \wedge (p \vee r)$ & \multirow{2}{*}{Leyes de distributivas} \\
			$p \wedge (q \vee r) \equiv (p \wedge q) \vee (p \wedge r)$ & \\
			\hline
			$(p \wedge q)' \equiv p' \vee q'$ & \multirow{2}{*}{Leyes de Morgan} \\
			$(p \vee q)' \equiv p' \wedge q'$ & \\
			\hline
			$p \vee (p \wedge q) \equiv p$ & \multirow{2}{*}{Leyes de absorcion} \\
			$p \wedge (p \vee q) \equiv p$ & \\
			\hline
			$p\vee p' \equiv V$ & \multirow{2}{*}{Leyes de negacion} \\
			$p\wedge p' \equiv F$ & \\
			\hline
		\end{tabular}
	\end{center}}
\quad
	\parbox{.5\linewidth}{
	\label{tab:equivalencias-logicas-implicaciones}
	\begin{center}
		\begin{tabular}{|c|}
			\hline
			\textit{Equivalencia}\\
			\hline
			$p \rightarrow q \equiv p' \vee q$\\
			$p \rightarrow q \equiv q' \vee p$\\
			$p \vee q \equiv p' \rightarrow q$\\
			$p \wedge q \equiv (p \vee q')'$\\
			$(p \vee q') \equiv p \rightarrow q'$\\
			$(p \rightarrow q) \wedge (p \rightarrow r) \equiv p \rightarrow (q \wedge r)$\\
			$(p \rightarrow r) \wedge (q \rightarrow r) \equiv (p \vee q) \rightarrow r$\\
			$(p \rightarrow q) \vee (p \rightarrow r) \equiv p \rightarrow (q \vee r)$\\
			$(p \rightarrow r) \vee (q \rightarrow r) \equiv (p \wedge q) \rightarrow r$\\
			\hline
			$p \leftrightarrow q \equiv (p \rightarrow q) \wedge (q \rightarrow p)$\\
			$p \leftrightarrow q \equiv p' \leftrightarrow q'$\\
			$p \leftrightarrow q \equiv (p \wedge q) \vee (p' \wedge q')$\\
			$(p \leftrightarrow q)' \equiv p \leftrightarrow q'$\\
			\hline
			
		\end{tabular}
	\end{center}}
\end{table}

\subsubsection*{Satisfaccion}

Una proposición compuesta es \textbf{satisfactoria} si hay una asignación de valores de verdad a sus variables que la hace verdadera (es decir, cuando es una tautología o una contingencia). Cuando no existen tales asignaciones, es decir, cuando la proposición compuesta es falsa para todas las asignaciones de valores de verdad a sus variables, la proposición compuesta es \textbf{git}. Nótese que una proposición compuesta no es satisfactoria si y solo si su negación es verdadera para todas las asignaciones de valores de verdad a las variables, es decir, si y solo si su negación es una tautología.

Cuando encontramos una asignación particular de valores de verdad que hace que una proposición compuesta sea verdadera, hemos demostrado que es satisfactoria; tal asignación se denomina solución de este problema de satisfacibilidad particular. Sin embargo, para mostrar que una proposición compuesta es insatisfactoria, necesitamos mostrar que toda asignación de valores de verdad a sus variables la hace falsa. Aunque siempre podemos usar una tabla de verdad para determinar si una proposición compuesta es satisfactoria, a menudo es más eficiente no hacerlo.

\subsection{Predicados y cuantificadores}

\subsubsection*{Predicados}

Declaraciones que involucran variables, como "$x > 3$", "$x = y + 3$", "La computadora $x$ funciona adecuadamente", son declaraciones que no son ni falsas ni verdaderas al no estar especificados los valores de las variables.

La declaracion "$x$ es mayor que 3" tiene 2 partes. La primera parte, la variable $x$, es el sujeto de la declaracion. La segunda parte, el \textbf{predicado}, "es mayor que 3", se refiere a la propiedad que el sujeto de la delcaracion puede tener.  Podemos denotar el enunciado "x es mayor que 3" por P (x), donde P denota el predicado "es mayor que 3" y x es la variable. También se dice que el enunciado P (x) es el valor de la \textbf{función proposicional} P en x. Una vez que se ha asignado un valor a la variable x, el enunciado P (x) se convierte en una proposición y tiene un valor de verdad.\\

\textbf{Ejemplo.} $P(x)$ denota la declaracion "$x > 3$". Cuales son los valores de verdad de $P(4)$ y $P(2)$?
Obtenemos la declaracion $P(4)$ setenado $x = 4$ en la delcaracion "$x > 3$". Por lo tanto, $P(4)$, cuya declaracion es "$4 > 3$", es verdadera. Sin embargo, $P(2)$ es claramente falsa.

\subsubsection*{Precondiciones y Poscondiciones}

Los predicados se usan para establecer exactitud en los programas informaticos, esto es, mostrar que los programas de computadoras siempre generan la salida deseada dada una entrada valida. Las declaraciones que describen una entrada valida se conocen como \textbf{precondiciones} y las condiciones que la salida deberia satisfacer cuando se corre el programa se conocen como \textbf{poscondiciones}.

\subsubsection*{Cuantificadores}

Cuando todas las variables de una funcion proposicional se le han asignado valores, la sentencia resultante se convierte en una proposicion con un cierto valor de verdad. No obstante, hay otra forma importante, llamada \textbf{cuantificacion}, de crear una proposicion a partir de una funcion proposicional. Tenemos la cuantificacion universal y la cuantificacion existencial.

\subsubsection*{El cuantificador universal}\label{sec:cuantificador-universal}

Muchas sentencias matematicas imponen que una propiedad es verdadera para todos los valors de una variable en un dominio particular, llamado el \textbf{universo de discurso} o \textbf{dominio}. Tales sentencias se expresan utilizando un cuantificador universal. La cuantificacion universal de una funcion proposicional es la proposicion que afirma que $P(x)$ es verdadera para todos los valores de $x$ en el dominio. EL dominio especifica los posibles valores de la variable $x$.\\

\textbf{Definicion.} La \textit{cuantificacion universal} de $P(x)$ es la proposicion $P(x)$ es verdadera para todos los valores $x$ del dominio. La notacion $\forall x P(x)$ denota la cuantificacion universal de $P(x)$. Aqui llamaremos al simbolo $\forall$ el \textbf{cuantificador universal}. La proposicion $\forall x P(x)$ se lee como \textit{para todo $x P(x)$} o \textit{para cada $x P(x)$} o \textit{para cualquier $x P(x)$}.\\

Para mostrar que una sentencia de la forma $\forall x P(x)$ es falsa, donde $P(x)$ es una funcion proposicional, solo necesitamos encontrar un valor de $x$ del dominio para el cual $P(x)$ sea falsa. Este valor de $x$ se llama \textbf{contraejemplo} de la sentencia $\forall x P(x)$.

\subsubsection*{El cuantificador existencial}\label{sec:cuantificador-existencial}

Muchas sentencias matematicas afirman que hay un elemento con una cierta propiedad. Tales sentencias se expresan mediante cuantificadores existenciales. Con un cuantificador existencial formamos una proposicion que es verdadera si y solo si $P(x)$ es verdadera para al menos un valor de $x$ en el dominio.\\

\textbf{Definicion.} La \textit{cuantificacion existencial} de $P(x)$ es la proposicion \textit{Existe un elemento $x$ en el dominio tal que $P(x)$ es verdadera.} Usamos la notacion $\exists x P(x)$ para la cuantificacion existencial de $P(x)$. El simbolo $\exists$ se denomina \textbf{cuantificador existencial}. La cuantificacion existencial $\exists x P(x)$ se lee como \textit{Hay un $x$ tal que $P(x)$} o \textit{Hay al menos un $x$ tal que $P(x)$} o \textit{Para algun $x P(x)$}

\begin{table}[H]
	\caption*{Cuantificadores}
	\begin{center}
		\begin{tabular}{|c|c|c|}
			\hline
			\textit{Sentencia} & \textit{¿Cuando es verdadera?} & \textit{¿Cuando es falsa?}\\
			\hline
			$\forall x P(x)$ & $P(x)$ es verdadera para todo $x$ & Hay un $x$ para el que $P(x)$ es falsa\\
			\hline
			$\exists x P(x)$ & Hay un $x$ para el que $P(x)$ es verdadera & $P(x)$ es falsa para todo $x$\\
			\hline
		\end{tabular}
	\end{center}
\end{table}

Cuando se quiere determinar el valor de verdad de una cuantificacion, a veces es util realizar una busqueda sobre todos los posibles valores del dominio. Supongamos que hay $n$ objetos en el dominio de la variable $x$. Para determinar si $\forall xP(x)$ es verdadera para todos ellos. Si encontramos un valor de $x$ para el cual $P(x)$ es falsa, habremos demostrado que $\forall xP(x)$ es falsa. En caso contrario, $\forall xP(x)$ es verdadera. Para ver si $\exists x P(x)$ es verdadera, barremos los $n$ posibles de $x$ buscando algun valor para el cual $P(x)$ sea verdadera. Si encontramos uno, entonces $\exists x P(x)$ es verdadera. Si no encontramos tal valor de $x$, habremos determinado que $\exists x P(x)$ es falsa.

\subsubsection*{El cuantificador de unicidad}

No hay limitación en el número de cuantificadores diferentes que podemos definir, como "hay exactamente dos", "no hay más de tres", "hay al menos 100", etc. De estos otros cuantificadores, el que se ve con más frecuencia es el cuantificador de unicidad, denotado por $\exists!$ o $\exists_{1}$. La notacion $\exists!xP(x)$ o $\exists_{1}xP(x)$ declara que "Existe un unico $x$ tal que $P(x)$ es verdadero". Por ejemplo $\exists!x(x - 1 = 0)$, donde el dominio es el conjunto de los numeros reales, declara que existe un unico numero real $x$ tal que $x - 1 = 0$. Esta es una declaracion verdadera, pues $x = 1$ es el unico numero real que cumple la declaracion.

\subsubsection*{Cuantificadores sobre dominios finitos}

Cuando el dominio de un cuantificador es finita, esto es, cuando todos sus elementos pueden ser listados, las declaraciones cuantificadas pueden expresarse usando la logica proposicional. En particular, cuando los elementos del dominio son $x_{1}, ..., x_{n}$, donde $n$ es un entero positivo, el cuantificador universal $\forall xP(x)$ es lo mismo que la conjuncion $P(x_{1})\wedge ... \wedge P(x_{n})$, porque la conjuncion es verdadera sii $P(x_{1})\wedge ... \wedge P(x_{n})$ son todos verdaderos.
Similarmente, cuando los elementos del dominio son $x_{1}, ..., x_{n}$, donde $n$ es un entero positivo, el cuantificador existencial $\exists xP(x)$ es lo mismo que la disjuncion $P(x_{1})\wedge ... \wedge P(x_{n})$ porque la disjuncion es verdadera sii al menos una de $P(x_{1}), ..., P(x_{n})$ es verdadera.

\subsubsection*{Conecciones entre cuantificadores y bucles}

A veces es de ayuda pensar en terminos de  bucle o busqeuda cuando se determina el valor de verdad de una cuantificacion. Suponer que hay $n$ objetos en el dominio para la variable $x$. Si $\forall xP(x)$ es verdadera, podemos iterar a traves de todos los $n$ valores de $x$ para ver si $P(x)$ es siempre verdadera. De otra forma, $\forall xP(x)$ es verdadera. Para ver si $\exists xP(x)$ es verdadero, iteramos a traves de los $n$ valores de $x$ buscando un valor para el cual $P(x)$ es verdadera. Si se encuentra, entonces $\exists xP(x)$ es verdadera, de otra forma, es falsa.

\subsubsection*{Cuantificadores con dominios restringidos}

A veces se usa una notacion abreviada para restringir el dominio de un cuantificador. En esta notacion, una condición que debe satisfacer una variable se incluye después del cuantificador.\\

\textbf{Ejemplos} 
\begin{enumerate}
	\item La declaracion $\forall x < 0 (x^{2} > 0)$ declara que para todo numero real $x$ con $x < 0, x^{2} > 0$. Esto dice, "El cuadrado de un numero real negativo es positivo". Esta declaracion es lo mismo que $\forall x(x < 0 \rightarrow x^{2} > 0)$.

	\item La declaracion $\forall y \neq 0(y^{3} \neq 0)$ declara que para todo numero real $y$ con $y\neq 0$, tenemos $y^{3} \neq 0$. Esto dice, "El cubo de todo numero real distinto de cero no es cero". Esta declaracion es lo mismo que $\forall y (y \neq 0 \rightarrow y^{3} \neq 0)$.

	\item Finalmente, la declaracion $\exists z > 0 (z^{2} = 2)$ dice que existe al menos un numero real $z$ con $z > 0$ tal que $z^{2} = 2$. Esto dicem "Hay una raiz cuadrada de 2 positiva". Esta declaracion es equivalente a $\exists z(z > 0 \wedge z^{2} = 2)$.
\end{enumerate}

Notese que la restriccion de un cuantificador universal es lo mismo que una cuantificacion universal de una declaracion condicional. Por otro lado, la restriccion de un cuantificador existencial es lo mismo que la cuantificacion existencial de una conjuncion.

\subsubsection*{Precedencia de Cuantificadores}

Los cuantificadores $\forall$ y $\exists$ tienen mayor precedencia que los operadores logicos del calculo proposicional. Por ejemplo, $\forall xP(x) \vee Q(x)$ es la disjuncion de $\forall xP(x)$ y $Q(x)$. En otras palabras, significa $(\forall xP(x)) \vee Q(x)$ en lugar de $\forall x(P(x) \vee Q(x))$.

\subsubsection*{Variables ligadas}

Cuando un cuantificador se usa sobre la variable $x$ o cuando asignamos un valor a esta variable, decimos que la variable aparece \textbf{ligada}. Una variable que no aparece ligada por un cuantificador o fijada a un valor particular, se dice que es \textbf{libre}. Todas las variables que aparecen en una funcion proposicional deben ser ligadas para convertirla en proposicion. Esto se puede hacer utilizando una combinacion de cuantificadores universales, cuantificadores existenciales y asignacion de valores.

La parte de una expresion logica a la cual se aplica el cuantificador se llama \textbf{alcance} de este cuantificador. Consecuentemente, una variable es libre si esta fuera del ambito de todos los cuantificadores en la proposicion compuesta.

\subsubsection*{Equivalencias logicas que involucran Cuantificadores}

\textbf{Definicion.} Las declaraciones que involucran predicados y cuantificadores se dicen \textit{logicamente equivalentes} sii tienen el mismo valor de verdad sin importar que predicados se sustituyen en estas declaracionies ni el dominio de discurso utilizado para las variables en estas funciones proposicionales. Usamos la notacion $S \equiv T$ para indicar que dos declaraciones $S$ y $T$ que involucran predicados y cuantificadores son logicamente equivalentes.

\subsubsection*{Negaciones}

Cuando el dominio de un predicado $P(x)$ consiste en $n$ elementos, donde $n$ es un entero positivo, las reglas de la negcion de sentencias cuantificadas son exactamente las mismas que las leyes de De Morgan. Esto es asi porque $(\forall x P(x))'$ es lo mismo que $(P(x_{1}) \wedge P(x_{2}) \wedge ... \wedge P(x_{n}))'$, equivalente a $P(x_{1})' \vee P(x_{2})' \vee ... \vee P(x_{n})'$ por las leyes de De Morgan. Esto es lo mismo que $\exists x P(x)'$. De forma analoga,  $(\exists x P(x))'$ es lo mismo que $(P(x_{1}) \vee P(x_{2}) \vee ... \vee P(x_{n}))'$, equivalente a $P(x_{1})' \wedge P(x_{2})' \wedge ... \wedge P(x_{n})'$ por las leyes de De Morgan, lo que equivale  $\forall x P(x)'$.

\begin{table}[H]
	\caption*{Cuantificadores}
	\begin{center}
		\begin{tabular}{|c|c|c|c|}
			\hline
			\textit{Negacion} & \textit{proposicion compuesta equivalente} & \textit{¿Cuando es verdadera la negacion?} & \textit{¿Cuando es falsa?}\\
			\hline
			$(\forall x P(x))'$ & $\forall x P(x)'$ & Para cada $x$, $P(x)$ es falsa & Hay un $x$ para el que $P(x)$ es verdadera\\
			\hline
			$(\exists x P(x))'$ & $\exists x P(x)'$ & Hay un $x$ para el que $P(x)$ es falsa & $P(x)$ es verdadera para cada $x$\\
			\hline
		\end{tabular}
	\end{center}
\end{table}

\subsubsection*{Premisas, Conclusiones, y Argumentos}

\textbf{Ejemplo.} Considere estas declaraciones. Las primeras dos son llamadas \textit{premisas} y la tercera es llamada \textit{conclusion}. El conjunto entero es llamado \textit{argumento}.\\
\textit{"Todos los leones son fieras."}\\
\textit{"Algunos leones no toman cafe."}\\
\textit{"Algunas criaturas fieras no toman cafe."}\\
Podemos expresar esas declaraciones como:\\
$\forall x(P(x) \rightarrow Q(x))$\\
$\exists x(P(x) \wedge R(x)')$\\
$\exists x(Q(x) \wedge R(x)')$\\

\subsection{Cuantificadores anidados}

Son cuantificadores que se localizan dentro del rango de aplicacion de otros cuantificadores, como en la sentencia $\forall x \exists y (x + y = 0)$. Los cuantificadores anidados se usan tanto en matematicas como en ciencais de la computacion. 

\subsubsection*{El orden de los cuantificadores}

\begin{table}[H]
	\caption*{Cuantificadores de dos variables}
	\begin{center}
		\begin{tabular}{|c|c|c|}
			\hline
			\textit{Sentencia} & \textit{¿Cuando es verdadera?} & \textit{¿Cuando es falsa?}\\
			\hline
			$\forall{x}\forall{y}P(x, y)$ & $P(x, y)$ es verdadera & Hay un par $x, y$ para el cual $P(x, y)$\\
			$\forall{x}\forall{y}P(x, y)$ & para todo $x, y$ & es falsa\\
			\hline
			$\forall{x}\exists{y}P(x, y)$ & Para todo $x$ hay un $y$ para el & Hay un $x$ tal que $P(x, y)$ \\
			 & cual $P(x, y)$ es verdadera & es falsa para todo $y$ \\
			\hline
			$\exists{x}\forall{y}P(x, y)$ & Hay un $x$ tal que $P(x, y)$ & Para todo $x$ hay un $y$ \\
			 & es verdadera para todo $y$ & para el cual  $P(x, y)$ es falsa \\
			\hline
			$\exists{x}\exists{y}P(x, y)$ & Hay un par $x, y$ para el cual & $P(x, y)$ es falsa para todo $x, y$\\
			$\exists{x}\exists{y}P(x, y)$ & $P(x, y)$ es verdadera &  \\
			\hline
		\end{tabular}
	\end{center}
\end{table}

\subsubsection*{Ejemplos}

\begin{enumerate}
	\item Sea $P(x, y)$ la proposicion $"x + y = y + x"$. Cuales son los valores de verdad de las cuantificaciones $\forall x \forall y P(x, y)$ y $\forall y \forall x P(x, y)$, si las variables son numeros reales?
	\textit{Solucion:}
	La sentencia
	\begin{equation*}
		\forall x\forall y P(x,y),
	\end{equation*}
	denota la proposicion
	\begin{center}
		"Para todos los numeros reales $x$, para todos los numeros reales $y$, $x + y = y + x$"
	\end{center}
	La sentencia
	\begin{equation*}
		\forall y\forall x P(x,y),
	\end{equation*}
	denota la proposicion
	\begin{center}
		"Para todos los numeros reales $y$, para todos los numeros reales $x$, $x + y = y + x$"
	\end{center}
	Ambas son equivalentes.

	\item $Q(x,y)$ denota $"x + y = 0"$. Cuales son los valores de verdad de la cuantificacion $\exists y \forall x Q(x,y)$ y $\forall x \exists y Q(x,y)$, donde el dominio son los reales?
	\textit{Solucion:}
	La cuantificacion
	\begin{equation*}
		\exists y \forall x Q(x,y),
	\end{equation*}
	denota la proposicion
	\begin{center}
		"Hay un numero real $y$ tal que para todo numero real $x$, $Q(x,y)$"
	\end{center}
	Sin importar el valor elegido de $y$, hay solo un valor de $x$ para que $x + y = 0$. Porque no hay numeor real $y$ tal que $x + y = 0$ para todo numero real $x$, la proposicion $\exists y \forall x Q(x,y)$ es falsa.
	La cuantificacion
	\begin{equation*}
		\forall x \exists y Q(x,y),
	\end{equation*}
	denota la proposicion
	\begin{center}
		"Para todo numero real $x$ hay un numero real $y$ tal que $Q(x,y)$."
	\end{center}
	Dado un numero real $x$, hay un numero real $y$ tal que $x + y = 0$; a saber, $y = -x$. Por lo tanto, la declaracion $\forall x \exists y Q(x,y)$ es verdadera.
\end{enumerate}

\subsubsection*{Traducir oraciones en español a expresiones logicas}

\subsubsection*{Ejemplos}

\begin{enumerate}
	\item Supongamos que el dominio de las variables reales $x$ e $y$ consiste en todos los reales. La sentencia 
	\begin{equation*}
		\forall{x} \forall{y} (x + y = y + x)
	\end{equation*} 
	afirma que $x + y = y + x$ para todo par de numeros reales $x$ e $y$. Es la ley conmutativa para la suma de los numeros reales. De la misma forma la sentencia 
	\begin{equation*}
		\forall{x} \exists{y} (x + y = 0)
	\end{equation*} 
	afirma que para cada numero real $x$ hay un real $y$ tal que $x + y = 0$. Esto declara que todo numero real tiene un inverso para la suma. Analogamente, la sentencia 
	\begin{equation*}
		\forall{x}\forall{y}\forall{z} (x + (y + z) = (x + y) + z)
	\end{equation*} 
	es la ley asociativa para la suma de numeros reales.
	
	\item $\forall{x} \forall{y} ((x > 0) \wedge (y < 0) \rightarrow (xy < 0))$ $\equiv$ todo numero $x$ positivo y para todo $y$ negativo, el producto de ese numero $x$ por $y$ sera menor a cero.
	
	\item $\forall{x} (C(x) \vee \exists{y} (C(y) \wedge F(x, y)))$ $\therefore$ \textit{C(x) = x tiene una computadora}, \textit{F(x, y) = x e y son amigos} y el dominio tanto para $x$ como para $y$ consiste en todos los estudiantes de tu facultad.
	\textit{Solucion:}
	Nos dice que para cada estudiante $x$ de tu facultad, $x$ tiene una computadora o hay un estudiante $y$ tal que $y$ tiene una computadora y $x$ e $y$ son amigos. Con otras palabras, todo estudiante de tu facultad tiene una computadora o un amigo que tiene una.
\end{enumerate}

\subsubsection*{Formalizacion de sentencias en expresiones logicas}

\subsubsection*{Ejemplos}

\begin{enumerate}
	\item Expresa la sentencia «Si una persona es del sexo femenino y tiene un hijo, esta persona es la madre de alguien» como una expresión lógica que involucre predicados, cuantificadores $-$cuyo dominio es el conjunto de todas las personas$-$ y conectivos lógicos.\\
	\textit{Solucion:} La frase anterior se puede expresar como «Para toda persona $x$, si la persona $x$ es del sexo femenino y la persona $x$ tiene un hijo, entonces existe una persona y tal que Ja persona $x$ es madre de la persona $y$». Introducimos los predicados $F(x)$ para representar «$x$ es del sexo femenino», $P(x)$ para representar «$x$ tiene un hijo» y $M(x, y)$ para representar «$x$ es madre de $y$» . La frase original se puede expresar como 
	\begin{equation*}
		\forall{x}((F(x) \wedge P(x)) \rightarrow \exists{y}M(x, y))
	\end{equation*}
	\begin{equation*}
		\forall{x}\exists{y}((F(x) \wedge P(x)) \rightarrow M(x, y))
	\end{equation*}
	
	\item «La suma de dos enteros positivos es positiva» $\equiv$ $\forall{x}\forall{y} ((x > 0) \wedge (y > 0) \rightarrow (x + y > 0))$
	
	\item «Todo numero real, excepto cero, tiene un inverso para el producto» $\equiv$ $\forall{x}\exists{y} ((x \neq 0) \rightarrow (xy = 1))$
	
	\item Definicion de limite usando cuantificadores
	
	\begin{equation*}
		\lim_{x \rightarrow a} f(x) = L
	\end{equation*}
	Tenemos que para todo numero real $\varepsilon > 0$, existe un numero real $\delta > 0$ tal que $|f(x) - L| < \varepsilon$ siempre que $0 < |x - a| < \delta$.
	\begin{equation*}
		\forall{\varepsilon}\exists{\delta} (0 < |x - a| < \delta \rightarrow |f(x) - L| < \varepsilon)
	\end{equation*}
	\begin{center}
		o
	\end{center}
	\begin{equation*}
		\forall{\varepsilon} > 0 \exists{\delta} > 0 \forall{x} (0 < |x - a| < \delta \rightarrow |f(x) - L| < \varepsilon)
	\end{equation*}
\end{enumerate}

\subsubsection*{Cuantificadores como bucles}

De forma similar para determinar si $\forall{x}\exists{y}P(x, y)$ es verdadera, recorreremos en un bucle todos los valores de $x$. Para cada $x$, recorremos en un bucle los valores de $y$ hasta que encontramos un $y$ para el cual $P(x, y)$ es verdadera. Si para todos los $x$ encontramos tal valor de $y$, entonces $\forall{x}\exists{y}P(x, y)$ es verdadera; si para algun $x$ no encontramos un valor de $y$ con esa propiedad, entonces $\forall{x}\exists{y}P(x, y)$ falsa.
Para ver si $\exists{x}\forall{y}P(x, y)$ es verdadera, recorremos los valores de $x$ en un bucle hasta que encontramos un $x$ para el cual $P(x)$ es siempre verdadera cuando recorremos en un bucle todos los valores de $y$. Una vez encontrado tal valor de $x$, sabemos que $\exists{x}\forall{y}P(x, y)$ es verdadera. Si no encontramos nunca un $x$ como ese, entonces sabremos que $\exists{x}\forall{y}P(x, y)$ es falsa.
Finalmente, para saber si $\exists{x}\exists{y}P(x, y)$ es verdadera, recorremos en un bucle los valores de $x$, y para cada valor de $x$ recorremos los valores de $y$ hasta que encontremos un $x$ oara el cual haya un $y$ que verifique que $P(x, y)$ sea verdadera.

\subsubsection*{Traducir enunciados matemáticos en enunciados que involucran cuantificadores anidados}

\subsubsection*{Ejemplos}

\begin{enumerate}
	\item Traducir la delcaracion "La suma de dos enteros positivos siempre es positiva" en una expresion logica:
	\textit{Solucion:}
	\begin{equation*}
		\forall x \forall y ((x > 0) \wedge (y > 0) \rightarrow (x + y > 0))
	\end{equation*}

	\item Traducir la declaracion "Todo numero real excepto el cero tiene un inverso multiplicativo":
	\textit{Solucion:}
	\begin{equation*}
		\forall x ((x \neq 0) \rightarrow \exists y(x\cdot y = 1))
	\end{equation*}
\end{enumerate}

\subsubsection*{Traducir cuantificadores anidados a español}

\subsubsection*{Ejemplos}

\begin{enumerate}
	\item Traducir la declaracion
	\begin{equation}
		\forall x(C(x) \vee \exists y(C(y) \wedge F(x,y)))
	\end{equation}
	a español, donde $C(x)$ es "$x$ tiene una computadora", $F(x,y)$ es "$x$ e $y$ son amigos", y el dominio de $x$ e $y$ son los estudiantes de tu escuela:

	\textit{Solucion:}
	La declaracion dice que para todo estudiante $x$ de mi escuela, $x$ tiene una computadora o hay un estudiante $y$ tal que $y$ tiene una computadora y $x$ e $y$ son amigos. En otras palabras, todo estudiante en mi escuela tiene una computadora o tiene un amigo que tiene una.

	\item Traducir la declaracion
	\begin{equation}
		\exists x \forall y \forall z ((F(x,y) \wedge F(x, z) \wedge (y \neq z)) \rightarrow F(y,z)')
	\end{equation}
	a español, donde $F(a,b)$ significa que $a$ y $b$ son amigos y el dominio de $x, y, x$ consiste en todos los estudiantes de tu escuela:

	\textit{Solucion:}
	Primero examinamos la expresion $(F(x,y) \wedge F(x, z) \wedge (y \neq z)) \rightarrow F(y,z)'$. Esta expresion dice que si los estudiantes $x$ e $y$ son amigos, los estudantes $x$ y $z$ son amigos, y ademas, si $y$ y $z$ no son el mismo estudiante, entonces $y$ y $z$ no son amigos. A esto sigue que la declaracion original, que es triplemente cuantificada, dice que hay un estudiante $x$ tal que para todos los estudiantes $y$ y todos los estudiantes $z$ distintos de $y$, si $x$ e $y$ son amigos y $x$ y $z$ son amigos, entonces $y$ y $z$ no son amigos. EN otras palabras, hay un estudiante cuyos amigos no son amigos entre si.
\end{enumerate}

\subsubsection*{Negacion de cuantificadores anidados}

Las sentencias con varios cuantificadores anidados se pueden negar aplicando sucesivamente las reglas de negación de las sentencias que contienen un único cuantificador.

\subsubsection*{Ejemplos}

\begin{enumerate}
	\item Negar $\forall{x}\exists{y} (xy = 1)$ de tal forma que ninguna negacion preceda al cuantificador.\\
	\textit{Solucion:}
	\begin{equation*}
		(\forall{x}\exists{y} (xy = 1))' \equiv \forall{x}(\exists{y} (xy = 1))' \equiv \forall{x}\exists{y} (xy = 1)'
	\end{equation*}
	\begin{equation*}
		\therefore (xy = 1)' \equiv (xy\neq1)
	\end{equation*}
	\begin{equation*}
		\Rightarrow \forall{x}\exists{y} (xy\neq1)
	\end{equation*}
	
	\item Usa cuantificadores para expresar la sentencia «No existe ninguna mujer que haya viajado en un vuelo de cada una de las líneas aéreas del mundo».\\
	\textit{Solucion:}
	\begin{equation}
		\begin{aligned}
			(\forall{w}\forall{a}\exists{f} (P(w, f) \wedge Q(f, a)))' &\equiv \forall{w}'\forall{a}\exists{f} (P(w, f) \wedge Q(f, a))\\
			&\equiv \forall{w}\forall{a}'\exists{f} (P(w, f) \wedge Q(f, a))\\
			&\equiv \forall{w}\forall{a}\exists{f}' (P(w, f) \wedge Q(f, a))\\
			&\equiv \forall{w}\forall{a}\exists{f} (P(w, f) \wedge Q(f, a))'\\
			&\equiv \forall{w}\forall{a}\exists{f} (P(w, f)' \wedge Q(f, a)')
		\end{aligned}
	\end{equation}
	«Para toda mujer hay una línea aérea tal que, para todo vuelo,
	esta mujer no ha viajado en ese vuelo o ese vuelo no es de esa línea aérea».

	\item Use cuantificadores y predicados para expresar el hecho de que $\lim_{x \rightarrow a} f(x)$ no existe cuando $f(x)$ es una funcion de valor real de variable real $x$ y $a$ pertenece al dominio de $f$
	\textit{Solucion:}
	Decir que tal limite no existe significa que para todo numero real $L$, $\lim_{x \rightarrow a} f(x) \neq L$.
	\begin{equation*}
		\begin{aligned}
			(\forall \varepsilon > 0 \exists \sigma > 0 \forall x(0 < |x - a| < \sigma \rightarrow |f(x) - L | < \varepsilon)' &\equiv \exists \varepsilon > 0 (\exists \sigma > 0 \forall x(0 < 1x - a| < \sigma \rightarrow |f(x) - L | < \varepsilon))'\\
			&\equiv \exists \varepsilon > 0 \forall \sigma > 0 (\forall x(0 < |x - a| < \sigma \rightarrow |f(x) - L | < \varepsilon)'\\
			&\equiv \exists \varepsilon > 0 \forall \sigma > 0 \exists x(0 < |x - a| < \sigma \rightarrow |f(x) - L | < \varepsilon)'\\
			&\equiv \exists \varepsilon > 0 \forall \sigma > 0 \exists x(0 < |x - a| < \sigma \wedge |f(x) - L | \geq \varepsilon)
		\end{aligned}
	\end{equation*}
	\begin{equation*}
		\Rightarrow \forall L \exists \varepsilon > 0 \forall \sigma > 0 \exists x(0 < |x - a| < \sigma \wedge |f(x) - L | \geq \varepsilon)
	\end{equation*}
\end{enumerate}

\subsection{Reglas de inferencia}

Las pruebas en matematica son argumentos validos que establecen la validez de una declaracion matematica. Por \textbf{argumento}, nos referimos a una secuencia de declaraciones que terminan en una conclusion. Por \textbf{valido} nos referimos a que la conclucion, o declaracion final del argumento, debe deducirse de la veracidad de las afirmaciones o \textbf{premisas} precedentes del argumento. Es decir, un argumento es valido sii es imposible para todas las premisas verdaderas llegar a una conclusion falsa. Para deducir nuevas declaraciones de declaraciones previas, debemos usar las reglas de inferencia las cuales son plantillas para la construccion de argumentos validos. Las reglas de inferencia son nuestras herramientas basicas para establecer la validez de declaraciones.

\subsubsection*{Argumentos validos en logica proposional}

\textbf{Definicion.} Un \textit{argumento} en logica proposicional es una secuencia de proposiciones. Todas menos la ultima proposicion en el argumento se llaman \textit{premisas} y la proposicion final se llama \textit{conclusion}. Un argumento es \textit{valido} si la validez de todas sus premisas implica que la conclusion sea verdadera.
Una \textit{forma de argumento} en logica proposicional es una secuencia de proposiciones compuestas que involucran variables proposicionales. Una forma de argumento es \textit{valida} si no importa qué proposiciones particulares sustituyan a las variables proposicionales en sus premisas, la conclusión es verdadera si todas las premisas son verdaderas.\\

\textbf{Observacion.} De la definicion de forma de argumento valido vemos que la forma de argumento con las premisas $p_{1}, ..., p_{n}$ y conclusion $q$ es valida exactamente cuando $(p_{1} \wedge ... \wedge p_{n}) \rightarrow q$ es una tautologia.
La clave para demostrar que un argumento en logica proposicional es valido es demostrar que su forma de argumento es valida.

\subsubsection*{Reglas de inferencia en logica proposicional}

\begin{table}[H]
	\caption*{Reglas de Inferencia}
	\begin{center}
		\begin{tabular}{|c|c|c|}
			\hline
			\textbf{\textit{Reglas de Inferencia}} & \textbf{\textit{Tautologia}} & \textbf{\textit{Nombre}}\\
			\hline
			$p$ &  & \\
			$p \rightarrow q$ & $(p\wedge (p \rightarrow q)) \rightarrow q$ & Modus ponens\\
			$\therefore q$ &  & \\
			\hline
			$q'$ &  & \\
			$p \rightarrow q$ & $(q' \wedge (p \rightarrow q)) \rightarrow p'$ & Modus tollens\\ 
			$\therefore p'$ &  & \\
			\hline
			$p \rightarrow q$ &  & \\
			$q \rightarrow r$ & $((p \rightarrow q) \wedge (q \rightarrow r)) \rightarrow (p \rightarrow r)$ & Silogismo hipotetico\\
			$\therefore p \rightarrow r$ &  & \\
			\hline
			$p \vee q$ &  & \\
			$p'$ & $((p \vee q) \wedge p') \rightarrow q$ & Silogismo disjuntivo\\
			$\therefore q$ &  & \\
			\hline
			$p$ & $p \rightarrow (p \vee q)$ & Adicion\\
			$\therefore p \vee q$ &  & \\
			\hline
			$p \wedge q$ & $(p \wedge q) \rightarrow p$ & Simplificacion\\
			$\therefore p$ &  & \\
			\hline
			$p$ &  & \\
			$q$ & $((p) \wedge (q)) \rightarrow (p \wedge q)$ & Conjuncion\\
			$\therefore p \wedge q$ &  & \\
			\hline
			$p \vee q$ &  & \\
			$p' \vee r$ & $((p \vee q) \wedge (p' \vee r)) \rightarrow (q \vee r)$ & Resolucion\\
			$\therefore q \vee r$ &  & \\
			\hline
		\end{tabular}
	\end{center}
\end{table}

\subsubsection*{Ejemplos}

\begin{enumerate}
	\item Indique qué regla de inferencia es la base del siguiente argumento: “Ahora está bajo cero y está lloviendo. Por lo tanto, ahora está bajo cero ".\\
	\textit{Solucion:} sea $p$ la proposicion "Ahora esta bajo cero", y sea $q$ la proposicion "Esta lloviendo ahora". El argumento es de la forma:
	\begin{equation*}
		\begin{aligned}
			&p \wedge q\\
			\therefore &p
		\end{aligned}
	\end{equation*}
	Este argumnto usa la regla de simplificacion.

	\item Indique qué regla de inferencia es la base del siguiente argumento:
	"Si hoy llueve, entonces no haremos asado hoy. Si no hacemos asado hoy, entonces haremos asado mañana. Por lo tanto, si llueve hoy, entonces haremos asado mañana.\\
	\textit{Solucion:} Sea $p$ la proposicion "Hoy esta lloviendo", sea $q$ la proposicion "No haremos asado hoy", y sea $t$ la proposicion "Haremos asado mañana". Entonces el argumento es de la forma:
	\begin{equation*}
		\begin{aligned}
			&p \rightarrow q\\
			&q \rightarrow r\\
			\therefore &p \rightarrow r
		\end{aligned}
	\end{equation*}
	Por lo tanto, el argumento es un silogismo hipotetico.
\end{enumerate}

\subsubsection*{Usando reglas de inferencia para construir argumentos}

\subsubsection*{Ejemplos}

\begin{enumerate}
	\item Mostrar que las premisas "Esta tarde no esta soleada y hace mas frio que ayer", "Nosotros iremos a nadar solo si esta soleado", "Si nosotros no vamos a nadar, entonces nosotros tomaremos un viaje en canoa", y "Si nosotros tomamos un viaje en canoa, entonces nosotros estaremos en casa para el anochecer" llevan a la conclusion de que "nosotros estaremos en casa al atardecer"\\
	\textit{Solucion:} Sean $p, q, r, s, t$ las proposiciones respectivas anteriores (en orden). ENtonces las premisas son $p' \wedge q$, $r \rightarrow p$, y $s \rightarrow t$. La conclusion es simplemente $t$. Necesitamos dar un argumento valido con tales premisas y conclusion. Construimos un argumento para mostrar que nuestras premisas llegan a la conclusion deseada:

	\begin{table}[H]
		\begin{center}
			\begin{tabular}{ c c c }
				& \textbf{Paso} & \textbf{Razon}\\
				$1.$ & $p' \wedge q$ & Premisa\\
				$2.$ & $'p$ & Simplificacion usando (1)\\
				$3.$ & $r \rightarrow p$ & Premisa\\
				$4.$ & $'r$ & Modus tollens usando (2) y (3)\\
				$5.$ & $'r \rightarrow s$ & Premisa\\
				$6.$ & $s$ & Modus ponens usando (4) y (5)\\
				$7.$ & $s \rightarrow t$ & Premisa\\
				$8.$ & $t$ & Modus ponens usando (6) y (7)
			\end{tabular}
		\end{center}
	\end{table}
	Notese que si se hubiese querido resolver con tabla de verdad, la misma deberia tener 32 filas.

	\item Mostrar que las premisas "Si tu me envias un e-mail, entonces yo terminare de escribir el programa", "Si tu no me envias un e.mail, entonces me ire a dormir temprano", y "Si yo voy a dormir temprano, entonces yo voy a despertarme sintiendome refrescado" llevan a la conclusion "Si yo no termino de escribir el programa, entonces me despertare sintiendome refrescado".
	\textit{Solucion:}
	\begin{table}[H]
		\begin{center}
			\begin{tabular}{ c c c }
				& \textbf{Paso} & \textbf{Razon}\\
				$1.$ & $p \rightarrow q$ & Premisa\\
				$2.$ & $q' \rightarrow p'$ & Contrapositiva de (1)\\
				$3.$ & $p' \rightarrow r$ & Premisa\\
				$4.$ & $q' \rightarrow r$ & Silogismo hipotetico de (2) y (3)\\
				$5.$ & $r \rightarrow s$ & Premisa\\
				$6.$ & $q' \rightarrow s$ & Silogismo hipotetico de (4) y (5)
			\end{tabular}
		\end{center}
	\end{table}
\end{enumerate}

\subsubsection*{Resolucion}

Los programas de computadoras se desarrollaron para automatizar tareas de razonamiento y demostrar teoremas. Muchos de estos programas una regla de inferencia conocida como \textbf{resolucion}. Esta regla de inferencia esta basada en la tautologia $((p \vee q) \wedge (p' \vee r)) \rightarrow (q \vee r)$. La disjuncion final en la regla de resolucion $q \vee r$, es llamada \textbf{resolvente}. Cuando $q = r$ en esta tautologia, obtenemos $(p \vee q) \wedge (p' \vee q) \rightarrow q$. Ademas, cuando $r = \textbf{F}$, obtenemos $(p \vee q) \wedge ('p) \rightarrow q$ (porque $q \vee \textbf{F} \equiv q)$, lo que es la tautologia en la cual la regla del silogismo disjuntivo esta basada.

Para construir pruebas en logica proposicional usando resolucion como la unica regla de inferencia, la hipotesis y la conclusion deben ser expresadas como \textbf{clausulas}, donde una clausula es una disjuncion de variables o negaciones de esas variables. Podemos reemplazar una declaracion en logica proposicional que no es una clausula por una o mas declaraciones equivalentes que si lo son. Por ejemplo supone que tenemos una declaracion de la forma $p \vee (q \wedge r)$. Porque $p \vee (q \wedge r) \equiv (p \vee q) \wedge (p \vee r)$, podemos reemplazar la declaracion $p \vee (q \wedge r)$ por dos declaraciones $p \vee q$ y $p \vee r$, cada una siendo una clausula. Podemos reeemplazar la declaracion de la forma $(p \vee q)'$ por las dos declarcaciones $p'$ y $q'$ porque la ley de De Morgan nos dice que $(p \vee q)' \equiv p' \wedge q'$. Ademas podemos reemplazar una declacacion condicional (impliacion) $p \rightarrow q$ por la disjuncion equivalente $p' \vee q$. 

\subsubsection*{Ejemplos}

\begin{enumerate}
	\item Use resolucion para mostrar que la hipostesis "Jasmine esta esquiando o no esta nevando" y "Esta nevando o Bart esta jugando hockey" implican que "Jasmine esta esquiando o Bart esta jugando hockey".\\
	\textit{Solucion:} Sea $p$ la proposicion "Esta nevando", $q$ la proposicion "Jasmine esta esquiando", y $r$ la proposicion "Bart esta jugando al hockey". Podemos representar la hipostesis como $p' \vee q$ and $p \vee r$, respectivamente. Usando resolucion, la proposicion $q \vee r$, "Jasmine esta esquiando o Bart esta jugando al hockey", cumple.

	\item Muestre que las premisas $(p \wedge q) \vee r$ y $r \rightarrow s$ implican la conclusion $p \vee s$.\\
	\textit{Solucion:} Podemos reescribir las premisas $(p \wedge q) \vee r$ como dos clausulas, $p \vee r$ y $q \vee r$. Ademas podemos reemplazar $r \rightarrow s$ por la clausula equivalente $r' \vee s$. Usando las dos clausulas $p \vee r$ y $r' \vee s$, podemos usar resolucion y concluir $p \vee s$.
\end{enumerate}

\subsubsection*{Falacias}

Varias falacias comunes surgen en argumentos incorrectos. Estas falacias se asemejan a las reglas de inferencia, pero se basan en contingencias más que en tautologías. Estos se analizan aquí para mostrar la distinción entre razonamiento correcto e incorrecto.

La proposicion $((p \rightarrow q) \wedge q) \rightarrow p$ no es una tautologia, porque es falsa cuando $p$ es falsa y $q$ es verdadera. De todas formas, hay muchos argumentos incorrectos que tratan esto como una tautologia. En otras palabras, tratan a los argumentos con premisas $p \rightarrow q$ y $q$ y conclusion $p$ como forma de argumento valida, lo cual no lo es. Este tipo de razonamiento incorrecto se llama la \textbf{falacio de afirmacion de conclusion}.

La proposicion $((p \rightarrow q) \wedge p') \rightarrow q'$ no es una tautologia, porque es falsa cuando $p$ es falsa y $q$ es verdadera. Muchos argumentos incorrectos usan esto incorrectamente como una regla de inferencia. Este tipo de razonamiento incorrecto se llama \textbf{falacia de la negacion de hipotesis}.

\subsubsection*{Ejemplos}

\begin{enumerate}
	\item Es el siguiente argumento valido? "Si tu realizas todos los problemas de este libro, entonces tu aprenderas matematica discreta. Tu aprendiste matematica discreta. Por lo tanto, tu realizaste todos los problemas en este libro.\\
	\textit{Solucion:} Sea $p$ la proposicion "Tu realizaste todos los problemas en este libro". Sea $q$ la proposicion "Tu aprendendiste matematica discreta". Entonces el argumento es de la forma: si $p \rightarrow q$ y $q$, entonces $p$. Este es un ejemplo de un argumento incorrecto utilizando la falacia por afirmacion de conclusion. En efecto, es posible para ti aprender matematica discreta de alguna forma distinta a realizar todos los problemas de este libro. (Puedes aprender matematica discreta leyendo, escuchando lecturas, haciendo algunos, pero no todos los problemas de este libro, etc).

	\item Sea $p$ y $q$ tal que el ejemplo anterior. Si la declaracion condicional $p \rightarrow q$ es verdadera, y $p'$ es verdadera, ¿es correcto concluir que $q'$ es verdadera? En otras palabras, es correcto asumir que tu no aprenderas matematica discreta si tu no resuelves cada problema de este libro, asumiendo que si tu realizas cada problema de este libro, entonces aprenderas matematica discreta?\\
	\textit{Solucion:} es posible que tu hayas aprendido matematica discreta aun si tu no realizar cada problema en este libro. Este argumento incorrecto es de la forma $p \rightarrow q$ y $p'$ implica $q'$, lo cual es un ejemplo de la falacia de la negacion de hipotesis.
\end{enumerate}

\subsubsection*{Reglas de inferencia para declaraciones cuantificadas}

Estas reglas de inferencia se usan extensivamente en los argumentos matematicos, a veces sin ser explicitamente mencionados.\\

La \textbf{instanciacion universal} es la regla de inferencia usada para concluir que $P(c)$ es verdadera, donde $c$ es un miembro particular de un dominio, dada una premisa $\forall xP(x)$. La instanciacion universal se usa cuando concluimos de una declaracion "Todas las mujeres son sabias" que "Lisa es sabia", donde Lisa es el miembro del dominio de las mujeres.

La \textbf{generalizacion universal}  es la regla de inferencia que declara que $\forall xP(x)$ es verdadera, dada una premisa de que $P(c)$ es verdadera para todos los elementos $c$ del dominio. La generalizacion universal es usada cuando mostramos que $\forall x P(x)$ es verdadero tomando un elemento arbitrario $c$ del dominio y mostrando que $P(c)$ es verdadera. El elemento $c$ que elegimos debe ser arbitrario, y no uno especifico. Cuando acertamos de $\forall x P(x)$ la existencia del elemento $c$ en el dominio, no tenemos control sobre $c$ y no podemos hacer otras suposiciones sobre $c$ distintas de las que vienen del dominio. La generalización universal se usa implícitamente en muchas pruebas en matemáticas y rara vez se menciona explícitamente. Sin embargo, el error de agregar suposiciones injustificadas sobre el elemento arbitrario $c$ cuando se usa la generalización universal es demasiado común en el razonamiento incorrecto.

La \textbf{instanciacion existencial} es la regla que nos premite concluir que hay un elemento $c$ en el dominio para el cual $P(c)$ es verdadera si sabemos que $\exists x P(x)$ es verdadera. Nosotros no podemos elegir un valor arbitrario de $c$ aqui, pero debe ser un $c$ para el cual $P(c)$ sea verdadera. Usualmente no tenemos conocimiento de tal $c$, solo sabemos que existe. Porque existe, le damos un nombre y continuamos nuestro argumento.

La \textbf{generalizacion existencial} es la regla de inferencia se usa para concluir que $\exists x P(x)$ es verdadera cuando se conoce un elemento particular $c$ con $P(c)$ verdadera. Esto es, si sabemos que un elemento $c$ en el dominio para el cual $P(c)$ es verdadera, entonces sabemos que $\exists x P(x)$ es verdadera.

\begin{table}[H]
	\caption*{Reglas de Inferencia para Declaraciones Cuantificadas}
	\begin{center}
		\begin{tabular}{| c | c |}
			\hline
			\textbf{\textit{Reglas de Inferencia}} & \textbf{\textit{Nombre}}\\
			\hline
			$\forall xP(x)$  & Instanciacion Universal\\
			$\therefore P(c)$ & \\
			\hline
			$P(c)$ para un $c$ arbitrario & Generalizacion universal\\
			$\therefore \forall xP(x)$  & \\
			\hline
			$\exists xP(x)$ & Instanciacion Existencial\\
			$\therefore P(c)$ para algun elemento $c$ & \\
			\hline
			$P(c)$ para algun elemento $c$ & Generalizacion Existencial\\
			$\therefore \exists xP(x)$ & \\
			\hline
		\end{tabular}
	\end{center}
\end{table}

\subsubsection*{Ejemplos}

\begin{enumerate}
	\item Mostrar que ls premisas "Todos en la clase de matematica discreta hicieron un curso de ciencias de la computacion" y "Marla es estudiante de esta clase" implican la conclusion "marla hizo un curso de ciencias de la computacion".\\
	\textit{Solucion:} $D(x)$ denota $x$ esta en esta clase de matematica discreta", y $C(x)$ denota "$x$ hizo un curso de ciencias de la computacion". Entonces las premisas son $\forall x (D(x) \rightarrow C(x))$ y $D(Marla)$. La conclucion es $C(Marla)$.
	Los siguientes pasos pueden ser usados para establecer la conclusion desde las premisas:
	\begin{table}[H]
		\begin{center}
			\begin{tabular}{ c c c }
				& \textbf{Paso} & \textbf{Razon}\\
				$1.$ & $\forall x(D(x) \rightarrow C(x))$ & Premisa\\
				$2.$ & $D(Marla) \rightarrow C(Marla)$ & instanciacion universal de (1)\\
				$3.$ & $D(Marla)$ & Premisa\\
				$4.$ & $C(Marla)$ & Modus ponens de (2) y (3)
			\end{tabular}
		\end{center}
	\end{table}

	\item Mostrar que las premisas "Un estudiante de esta clase no leyo el libro", y "Todos en esta clase aprobaron el examen" implican la conclusion "Alguien que paso el examen no ha leido el libro".\\
	\textit{Solucion:} $C(x)$ denota "$x$ esta en esta clase", $B(x)$ denota "$x$ ha leido el libro", y $P(x)$ denota "$x$ aprobo el primer examen". Las premisas son $\exists x (C(x) \wedge B(x)')$ y $\forall x(C(x) \rightarrow P(x))$. La conclusion es $\exists x(P(x) \wedge B(x)')$:
	\begin{table}[H]
		\begin{center}
			\begin{tabular}{ c c c }
				& \textbf{Paso} & \textbf{Razon}\\
				$1.$ & $\exists x(C(x) \wedge B(x)')$ & Premisa\\
				$2.$ & $C(a) \wedge B(a)'$ & Instanciacion existencial de (1)\\
				$3.$ & $C(a)$ & Simplificacion de (2)\\
				$4.$ & $\forall x(C(x) \rightarrow P(x))$ & Premisa\\
				$5.$ & $C(a) \rightarrow P(a)$ & Instanciacion universal de (4)\\
				$6.$ & $P(a)$ & Mdus ponens de (3) y (5)\\
				$7.$ & $B(a)'$ & Simplificacion de (2)\\
				$8.$ & $P(a) \wedge B(a)'$ & Conjuncion de (6) y (7)\\
				$9.$ & $\exists x(P(x) \wedge B(x)')$ & Generalizacion existencial de (8)
			\end{tabular}
		\end{center}
	\end{table}
\end{enumerate}

\subsubsection*{Combinando reglas de inferencia para proposiciones y declaraciones cuantificadas}

En los ejemplos anteriores se utiliza el modus ponens, una regla de inferencia para logica proposicional, y la instanciacion universal, una regla de inferencia para declaraciones cuantificadas. Su combinacion se llama \textbf{modus ponens universal}. Esta regla nos dice que si $\forall x (P(x) \rightarrow Q(x))$ es verdadera, y si $P(a)$ es verdadera para un elemento particular $a$ en el dominio del cuantificador universal, entonces $Q(a)$ debe ser verdadera. Por ejemplo:
\begin{equation*}
	\begin{aligned}
		&\forall x(P(x) \rightarrow Q(x))\\
		&P(a)\text{, donde $a$ es un elemento particular del domino}\\
		\therefore &Q(a)
	\end{aligned}
\end{equation*}

\subsubsection*{Ejemplo}

Asumir que "Para todo los enteros positivos $n$, si $n$ es mayor que 4, entonces $n^{2}$ es menor que $2^{n}$" es verdadero. Use el modus ponens universal para mostrar que $100^{2} < 2^{100}$.\\
\textit{Solucion:} $P(n)$ denota "$n > 4$" y $Q(n)$ denota "$n^{2} < 2^{n}$". La declaracion "Para todo los enteros positivos $n$, si $n$ es mayor que 4, entonces $n^{2}$ es menor que $2^{n}$" puede representarse como $\forall n(P(n) \rightarrow Q(n))$, donde el dominio son los enteros positivos. Asumimos que $\forall n(P(n) \rightarrow Q(n))$ es verdadero. Note que $P(100)$ es verdadero porque $100 > 4$. Por modus ponens univeral sabemos que $Q(100)$ es verdadero, porque $100^{2} < 2^{100}$.\\

Otra combinacion util de una regla de inferencia de logica proposicional y una regla de inferencia para declaraciones cuantificadas es el \textbf{modus tollens universal}. Este combina la instanciacion universal y el modus tollens y puede ser expresado:
\begin{equation*}
	\begin{aligned}
		&\forall x(P(x) \rightarrow Q(x))\\
		&Q(a)'\text{, donde $a$ es un elemento particular del domino}\\
		\therefore &P(a)'
	\end{aligned}
\end{equation*}
\end{document}